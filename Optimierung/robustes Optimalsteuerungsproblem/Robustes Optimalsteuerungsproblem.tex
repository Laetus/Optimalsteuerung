\documentclass[12pt,a4paper]{article}
\usepackage[ngerman]{babel}
\usepackage[T1]{fontenc}
\usepackage[utf8]{inputenc}

\usepackage{enumerate} %für beschriftete Nummerierung
\usepackage{multirow} %mehrere Zeilen zusammenfassen

\usepackage{makeidx} %fürs Stichwortverzeichnis

\usepackage{amsmath}
\usepackage{amssymb}
\usepackage{amsthm}
\usepackage{graphicx}
\usepackage{color}

\usepackage{makeidx}

\usepackage{hyperref} %links im inhaltsverzeichnis, muss als letztes stehen

\theoremstyle{definition}
\newtheorem{definition}{Definition}
\theoremstyle{plain}
\newtheorem{satz}{Satz}
	
\begin{document}
\title{Standardform des robusten Optimalsteuerungsproblems (Direct robust counterpart formulation) mit Slackvariablen}
\author{Annkathrin}
\maketitle

\tableofcontents

\section{Übersichtlichere Form}

\subsection*{Zielfunktion}
\begin{equation}
\min_{T, u(.), x(.), s_{0,0}, s_{0,1},..., s_{0, n_{f}},..., s_{n_{f},j}, D(.)} f_{0}(x(T), T)+e^{T}s_{0}
\end{equation}
Mit
\begin{itemize}
\item $f_{0}$: Zielfunktion aus Automodell (unseres ursprünglichen Problems)
	
\item $e^{T}=(1,1,...,1)\in\mathbb{R}^{n_{p}}$.

\item $s_{i,j}$: Slackvariablen, die benötigt werden, um Nichtdifferenzierbarkeit der Unendlichnorm (resultierend aus approximated robust counterpart, vgl. Diehl (11)-(13)) zu beseitigen. (vgl. Diehl (23)-(26) im diskreten Fall)

\item $D(.)$: Sensitivitäts-Matrix, die aus der numerisch effizienten Problemformulierung resultiert (vgl. Diehl (14)-(17) oder (31)-(32)).
\begin{equation*}
D(t)=\frac{dx(t)}{dx(0)}\frac{\partial x_{0}}{\partial p}(p)
\end{equation*}

\end{itemize}

\subsection*{Ungleichungsnebenbedingungen}
\begin{align}
f_{i}(x(t_{j}), t_{j})+e^{T}s_{i,j} &\leq 0 & \text{ for } i=1,...,n_{f}, j \text{ abhängig Gitter}\\
-AD(t_{j})^{T}\left(\frac{\partial f_{i}}{\partial x}(x(t_{j}), t_{j})\right)^{T}-s_{i,j}&\leq 0 &  \text{ for } i=1,...,n_{f}, j \text{ abhängig Gitter} \\
AD(t_{j})^{T}\left(\frac{\partial f_{i}}{\partial x}(x(t_{j}), t_{j})\right)^{T}-s_{i,j}&\leq 0 &  \text{ for } i=1,...,n_{f}, j \text{ abhängig Gitter}
\end{align}
Dabei kommem die zweite und dritte Ungleichungsserien von der oben genannten Slackformulierung (vgl. Diehl (23)-(26) im diskreten Fall).\\
\\
Mit:
\begin{itemize}
	\item $f_{i}$: Ungleichungsbedingungen, die unsere Straße und die Beschränkung der Steuerung beschreiben.
	
	\item $t_{j}$: Interpolationszeitpunkte der Straße/Gitterpunkte für die Ungleichungsbedingungen der Straße (eventuell gleich der Gitterpunkte bei der späteren Diskretisierung des Optimalsteuerungsproblems)
	
	\item $A$: Matrix aus der Beschreibung der für die Parametermenge verwendeten Höldernorm: $\| p \|=\|A^{-1}p\|_{\infty}$ (vgl. Diehl S.213 f.)
	\begin{equation*}
	A=diag\left(\frac{p_{u}-p_{l}}{2}\right)^{-1}
	\end{equation*} 
	
\end{itemize}

\subsection*{DGLs}
\begin{align}
\frac{dx(t)}{dt}&= G(x(t), u(t)) & \forall t\in [0,T]\\
\frac{ds_{i,j}}{dt}&= 0 & \forall t\in [0,T]\\
\frac{dD(t)}{dt}&= \frac{\partial G}{\partial x}(x(t), u(t))D(t) & \forall t\in [0,T]\\
x(0)&= x_{0}(p) &\\
s_{i,j}(0)&= s_{i,j} &\\
D(0)&= \frac{\partial x_{0}}{\partial p}(p) &
\end{align}
Dabei kommt die dritte DGL von der oben genannten numerisch effizienten Problemformulierung (vgl. Diehl (14)-(17) oder (31)-(32)).\\
Der zweite Satz DGLs wird eingeführt, damit das Problem in Standardform formuliert ist. Der numerische Integrator müsste sie als triviale DGLs erkennen.\\
\\
Mit:
\begin{itemize}
	\item $G$: Beschreibung der Dynamik des Autos.
	
\end{itemize}



\section{Resultierende Standardform}
Dafür schreibe die Matrix $D=\left(\begin{array}{c}
-D_{1}(t)-\\
-D_{2}(t)-\\
...\\
-D_{n_{p}(t)-}
\end{array}\right)$ als Spaltenvektor: $\left(\begin{array}{c}
D_{1}(t)\\
D_{2}(t)\\
...\\
D_{n_{p}}(t)
\end{array}\right)$.\\
Sei $z=(x(.), s_{0,0}, s_{0,1},..., s_{0, n_{f}},..., s_{n_{f},j}, D_{1}(.),..., D_{n_{p}}(.))^{T}$ der Zustandsvektor. Dann ist die Standardnormalform
\begin{equation}
\min_{T, u(.),z(.)} f_{0}(z_{1}(T), T)+e^{T}s_{0}
\end{equation}
Die Nebenbedingung und die DGL erhält man durch zusammenfassen von (2)-(4) bzw. von (5)-(10) zu einer Ungleichung bzw. DGL und durch Ersetzen der Zustände durch ihren jeweiligen $z$-Eintrag.\\ 

\textcolor{red}{\textbf{Achtung:}} $x$ besteht nicht mehr nur aus den Autozuständen $y,v,Lenkwinkel$, sondern wird pro Integral in der ursprünglichen Zielfunktion unserer Autobeschreibung und pro Parameter $p_{i}$ vergrößert. Letzteres müsste bereits in der Dynamikbeschreibung durch $G$ enthalten sein und auch ersteres müsste darin aufgenommen werden.


\end{document}