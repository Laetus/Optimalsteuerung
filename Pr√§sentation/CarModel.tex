
\begin{frame}
\frametitle{Model of Car}


\begin{figure}[T]
\centering
\includestandalone[angle = -10, width=0.8\textwidth]{cartikz}
\end{figure}





\note{
\begin{itemize}
\item Our car is described as a pointmass, for simplicity
\item Road is visualized in gray
\item Car as a rectangle
\item steering angle is denoted by $\beta(t)$
\begin{itemize}
\item We applied Newton's mechanic
\item The motion/ direction of motion is described with 
\begin{itemize}
\item The steering angle $\beta(t)$
\item The acceleration force pushing the car forward 
\item The central forces 
\item If the speed of the car is to high, it slides and may leave the road.
\end{itemize}
\end{itemize}
\end{itemize}
}

\end{frame}

\begin{frame}
\frametitle{Dynamical System}

\begin{block}{}
\begin{align*}
& \dot{x}(t) = G(x(t), u(t), p) & x(0) = x_0,
\end{align*}
where 
\begin{align*}
& x \in C^1([0, t_f], \mathbb{R}^{n}) \text{ is the state variable,}\\
& u \in C^1([0, t_f], \mathbb{R}^{m}) \text{ is the control variable,}\\
& p \in \mathbb{R}^{n_p} \text{ is the uncertain parameter vector.}
\end{align*}
\end{block}

\note{
\begin{itemize}
\item Motion of the car is described as a dynamical system, i.e., with a ODE and an initial value
\item $t_f$ is the final time
\item our states: path, velocity, steering angle
\item control: brake, motor torque (gas petal), steering angle velocity
\item parameter: weather, rolling friction coefficients, uncertain sea level due to wrong NAVI data
\end{itemize}
}
\end{frame}