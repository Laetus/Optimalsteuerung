

\begin{frame}
\frametitle{Problem Formulation}

\begin{block}{Discrete Optimization Problem}
	
\begin{align*}
 &&& \min_{x\in \mathbb{R}^{n_{x}}, u\in \mathbb{R}^{n_{u}}}  f_{0}(x, u)\\
& \text{ s.t. } &&  f_{i}(x,u) \leq 0 & \text{ for } & i=1,\ldots,n_{f}\\
&&&  g_{j}(x,u,p)=0 & \text{ for } & j=1,\ldots,n_{x}
%&\text{with} && p\in \mathbb{R}^{n_{p}} is an uncertain parameter vector.
\end{align*}
\begin{overprint}
\onslide<1>
\begin{equation*}
\text{with } p\in \mathbb{R}^{n_{p}} \text{ uncertain parameter vector}
\end{equation*}

\onslide<2->
\begin{equation*}
\text{with } p\in\mathbb{P}_{box}=\left\{\left. p\in\mathbb{R}^{n_{p}} \right| p_{lower}\leq p \leq p_{upper}\right\}
\end{equation*}
\end{overprint}
\end{block}


%\begin{block}{Constraint for Parameter Vector}
%
%\begin{equation*}
%p\in\mathbb{P}_{box}=\left\{\left. p\in\mathbb{R}^{n_{p}} \right| p_{lower}\leq p \leq p_{upper}\right\}
%\end{equation*}
\onslide<3->
Example:

%\begin{align*}
%
%& \text{height profil: } & \left\{\left. p \right| 0\leq p\leq 45\right\}
%\end{align*}


%\end{block}

\note{
I think this might not be clear: Both, the weather and height are $\in \mathbb{R}^{n_p}$ or not necessarily the same dimension? Later, you write $p in \mathbb{P}_{box}$
}

\end{frame}

\begin{frame}
\frametitle{Worst Case Formulation}

EXAMPLE?

%\begin{align*}
%\Phi_{i}(u):= & & \max_{x\in \mathbb{R}^{n_{x}}, p\in \mathbb{R}^{n_{p}}} f_{i}(x,u)\\
%	&\text{s.t. } & g(x,u,p)=0\\
%	& & p\in\mathbb{P}_{box}
%\end{align*}



%\begin{tabular}{ccl}
%$\Phi_{i}(u):=$ & & $ \max_{x\in \mathbb{R}^{n_{x}}, p\in \mathbb{R}^{n_{p}}} f_{i}(x,u)$\\
%&$\text{s.t. }$ & $g(x,u,p)=0$\\
%& & $p\in\mathbb{P}_{box}$
%\end{tabular}

\begin{align*}
\Phi_{i}(u):= &&& \max_{x\in \mathbb{R}^{n_{x}}, p\in \mathbb{R}^{n_{p}}} f_{i}(x,u)\\
& \text{s.t. } && g(x,u,p)=0\\
&&& p\in\mathbb{P}_{box}
\end{align*}

\begin{block}{Robust Counterpart}
\begin{align*}
&&&\min_{u\in\mathbb{R}^{n_{u}}} \Phi_{0}(u)\\
&\text{s.t. } &&\Phi_{i}(u)\leq 0 & \text{ for } & i=1,\ldots,n_{f}.
\end{align*}
$\Rightarrow$ bilevel structure!
\note{
I would not write the remark down.
}
\end{block}

\end{frame}

\begin{frame}
	\frametitle{Approximation Technique}

\begin{block}{Linearization}
	
	\begin{align*}
	\tilde{\Phi}_{i}(u):= &&& \max_{(x-\bar{x})\in\mathbb{R}^{n_{x}}, (p-\bar{p})\in\mathbb{R}^{n_{p}}} f_{i}(\bar{x}, u)+\frac{\partial f_{i}}{\partial x}(\bar{x}, u)(x-\bar{x}) \\
	& \text{s.t.} && \frac{\partial g}{\partial x}(\bar{x}, u, \bar{p})(x-\bar{x})+\frac{\partial g}{\partial p}(\bar{x}, u, \bar{p})(p-\bar{p})=0 \\
	&&& p-\bar{p} \text{ s.th. } p\in\mathbb{P}_{box}
	\end{align*}
	
	\note{
	Might not be clear what s.th. means	
	}
\end{block}	
	

	
\begin{block}	

\begin{align*}
&&&\min_{u\in\mathbb{R}^{n_{u}}, \bar{x}\in\mathbb{R}^{n_{x}}} \tilde{\Phi}_{0}(u)\\
&\text{s.t.} &&  \tilde{\Phi}_{i}(u)\leq 0 & \text{ for } & i=1,\ldots,n_{f}\\
&&& g(\bar{x}, u, \bar{p})=0
\end{align*}
\end{block} 
	
$\Rightarrow$ Standard Optimization Problem (SQP or else)

\note{
I would not write this commend down. (5 times 5 rule ;)
}
\end{frame}
