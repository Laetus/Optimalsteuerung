\documentclass[12pt,a4paper]{article}
\usepackage[ngerman]{babel}
\usepackage[T1]{fontenc}
\usepackage[utf8]{inputenc}

\usepackage{enumerate} %für beschriftete Nummerierung
\usepackage{multirow} %mehrere Zeilen zusammenfassen

\usepackage{makeidx} %fürs Stichwortverzeichnis

\usepackage{amsmath}
\usepackage{amssymb}
\usepackage{amsthm}
\usepackage{graphicx}

\usepackage{makeidx}

\usepackage{hyperref} %links im inhaltsverzeichnis, muss als letztes stehen

\theoremstyle{definition}
\newtheorem{definition}{Definition}
\theoremstyle{plain}
\newtheorem{satz}{Satz}
	
\begin{document}
	
\title{Parameter für unser Optimierungsproblem und Protokoll vom 13.05.2015}
\author{Sabina}
\maketitle

\noindent Datum: 11.05.2015\\
Uhrzeit: 12:00 - 14:00

%\tableofcontents

\section{Teilnehmer}
Annkathrin\\
Christian\\
Johannes\\
Sabina

\section{Modell}
\begin{enumerate}
\item 3 Unsicherheitsgrößen ohne Zeitabhängige Parameter
\begin{enumerate}
\item[(i)] Höhenprofil 1 Parameter a: $\alpha(t) = a\cdot H"ohenprofil$, $a\in[0.95,1.05]$
\item[(ii)] Wetter 3 Parameter die das Wetter beschreiben:\\0 = schlechtes Wetter z.B. Schnee,\\ 0.5 = mittel gutes Wetter z.B. Regen,\\1 = super Wetter, Sonnenschein\\
Wetter fließt in den $\mu_h$ ein, wobei $\mu_h$ angibt wann das Auto auf der Straße rutscht
\item[(iii)] Reibung 3 Parameter
\item[•] Insgesamt 7 zu bestimmende Parameter, was einer guten Größenordnung (5-10) entspricht
\end{enumerate}
\item makroskopisches Modell, d.h. lange Strecken werden betrachtet
\item Die Straße hat eine gewisse Breite und das Auto soll auf der Straße bleiben und sucht sich dort die optimale Linie
\item Ungleichungsnebenbedingungen im Allgemeinen vermeiden, insbesondere anfangs alle nichtlinearen Nebenbedingungen durch lineare ersetzen
\item alle Parameter für die Robustheit in die rechte Seite der Dynamik packen, d.h. sollten nicht in Ungleichungs- oder Gleichungsnebenbedingungen auftauchen
\item in den Ungleichungsnebenbedingungen die Abhängigkeit von T durch die einzelnen $t_i$ ersetzen und entsprechend auch in den Ableitungen 
\end{enumerate}

\section{Nächste Schritte}
\begin{enumerate}
\item Dynamik bestimmen, d.h. Ableitungen der rechten Seite der Dynamikgleichungen bestimmen (evtl. mithilfe von Mathematica) (Johannes)
\item robustes Modell formulieren (Annkathrin)
\item Multiple Shooting für die Diskretisierung der rechten Seite (Sabina)
\item einigen auf prodezurale oder objektorientiere Programmierung und Schnittstellen möglichst vorher definieren, insbesondere wenn Entscheidung für prodezurale Programmierung fällt 
\end{enumerate}

\section{Ausblick}
Falls sich in den Nebenbedingungen etwas zwischen 2 Diskretisierungspunkten ändert kann man das Gitter dort verfeinern und weitere Gitterpunkte einfügen um den Zeitpunkt zu erhalten, in dem die Änderung stattfindet
\end{document}	