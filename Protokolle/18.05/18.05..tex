\documentclass[12pt,a4paper]{article}
\usepackage[ngerman]{babel}
\usepackage[T1]{fontenc}
\usepackage[utf8]{inputenc}

\usepackage{enumerate} %für beschriftete Nummerierung
\usepackage{multirow} %mehrere Zeilen zusammenfassen

\usepackage{makeidx} %fürs Stichwortverzeichnis

\usepackage{amsmath}
\usepackage{amssymb}
\usepackage{amsthm}
\usepackage{graphicx}

\usepackage{makeidx}

\usepackage{hyperref} %links im inhaltsverzeichnis, muss als letztes stehen

\theoremstyle{definition}
\newtheorem{definition}{Definition}
\theoremstyle{plain}
\newtheorem{satz}{Satz}
	
\begin{document}
	
\title{Protokoll vom 18.05.2015}
\author{Christian}
\maketitle

Datum: 18.05.2015\\
Uhrzeit: 09:00 - 11:00

\tableofcontents

\section{Teilnehmer}
Annkathrin\\
Christian\\
Johannes\\
Sabina

\section{Midterm Präsentation}

Die zentralen Themen unseres Vortrags werden die folgenden Fragen sein:

\begin{itemize}
\item Was ist Optimalsteuerung?

\item Was ist Robustheit?
\end{itemize}
Zusätzlich wird Sabina etwas über "Multiple shooting" erzählen. Johannes und Annkathrin referieren über das Modell und robuste Optimalsteuerung und Christian über die Visualisierung.
Dabei werden wir, soweit das möglich ist, Beispiele hinzuziehen und auf Formeln verzichten. \\
Jeder wird seinen Teil in einer eigenen TeX-Datei anlegen die dann abschließend von Johannes zusammengeführt werden und mit Graphiken versehen werden.
Bis zur Präsentation soll auch schon das erste gerenderte Video eines Autos fertig sein.

\section{To Do}
\begin{itemize}
\item Fertigstellung des neuen, verbesserten Modells (Johannes)

\item Plausibilitätsüberprüfung des Modells (Alle außer Johannes)

\item Erstellen eines Polynoms (~Grad 10), das als Straße dient und Visualisierung dieser (Christian)

\item Test unseres Modells mit gewichteter Spritoptimierung (Sabina)

\item Die Zielfunktion soll von der Idee bleiben, aber das Integral muss ersetzt werden, in die Dynamik eingebunden werden

\item Slackformulierung der direct approximate robust counterpart formulation, Rücktrafo in Standardform (Annkathrin)
\end{itemize}

\section{Offenen Fragen}
Wie integriert man die $t_i$? (Nicht mehr T)
Können wir sie einfach einsetzen? (Nur noch Box Constraints und damit keine Zeitabhängigkeit mehr)

\end{document}	