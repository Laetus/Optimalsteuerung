\documentclass[12pt,a4paper]{article}
\usepackage[ngerman]{babel}
\usepackage[T1]{fontenc}
\usepackage[utf8]{inputenc}

\usepackage{enumerate} %für beschriftete Nummerierung
\usepackage{multirow} %mehrere Zeilen zusammenfassen

\usepackage{makeidx} %fürs Stichwortverzeichnis

\usepackage{amsmath}
\usepackage{amssymb}
\usepackage{amsthm}
\usepackage{graphicx}

\usepackage{makeidx}

\usepackage{hyperref} %links im inhaltsverzeichnis, muss als letztes stehen

\theoremstyle{definition}
\newtheorem{definition}{Definition}
\theoremstyle{plain}
\newtheorem{satz}{Satz}
	
\begin{document}
	
\title{Protokoll vom 13.05.2015}
\author{Sabina}
\maketitle

\noindent Datum: 27.05.2015\\
Uhrzeit: 11:30 - 14:00

%\tableofcontents

\section{Teilnehmer}
Annkathrin\\
Christian\\
Johannes\\
Sabina

\section{Modell}
\begin{itemize}
\item Krümmungsradius approximiert:\\
Der Krümmungsradius ist konstant auf der Breite der Straße  und entspricht dem Krümmungsradius der Mittellinie der Straße
\item Boxconstraints für v wählen, sodass v nicht negativ wird
\end{itemize}

\section{Allgemeines}
\begin{itemize}
\item Projektplan nicht in github sondern online updaten
\item Christian: main.pdf in git ignore hinzufügen
\end{itemize}

\section{Präsentation}
\begin{itemize}
\item Mathematik darf nicht fehlen
\item Problemstellung erklären und die Lösung dazu mathematisch ausführen
\item Mithilfe von Mindmap oder ähnlichem Struktur des Vortrags klar machen und zu diesem 'Inhaltsverzeichnis' auch innerhalb des Vortrags zurückkehren 
\end{itemize}

\end{document}	