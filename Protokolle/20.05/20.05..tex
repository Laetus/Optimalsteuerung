\documentclass[12pt,a4paper]{article}
\usepackage[ngerman]{babel}
\usepackage[T1]{fontenc}
\usepackage[latin1]{inputenc}

\usepackage{enumerate} %für beschriftete Nummerierung
\usepackage{multirow} %mehrere Zeilen zusammenfassen

\usepackage{makeidx} %fürs Stichwortverzeichnis

\usepackage{amsmath}
\usepackage{amssymb}
\usepackage{amsthm}
\usepackage{graphicx}

\usepackage{makeidx}

\usepackage{hyperref} %links im inhaltsverzeichnis, muss als letztes stehen

\theoremstyle{definition}
\newtheorem{definition}{Definition}
\theoremstyle{plain}
\newtheorem{satz}{Satz}
	
\begin{document}
	
\title{Protokoll vom 20.05.2015}
\author{Johannes}
\maketitle


Uhrzeit: 13:00 - 14:00

\tableofcontents

\section{Teilnehmer}
Annkathrin\\
Christian\\
Johannes\\
Sabina

\section{Midterm Presentation}
\begin{itemize}
\item Jeder erstellt zu seinem Teil entsprechende Folien
\item Johannes achtet auf eine einheitliche Gestaltung, insbesondere Erstellung der Graphiken
\item Skypgespraech am Wochenende.
\item Unter Umstaenden kann bereits ein Blendervideo gezeigt werden.
\end{itemize}

\section{Simulation}
\begin{itemize}
\item Besprechung von Ausgaben der Optimierung f�r verschiedene Aenderungen von Gewichtungsgroe�en.
\end{itemize}

\section{Robuste Formulierung}
\begin{itemize}
\item Wie modellieren wir Constraints hervorgerufen durch den Verlauf der Stra�e mit dem Ansatz von Diehl?
\item Annkathrin bespricht dies mit Sebastian.
\item Johannes bespricht finale Automodellierung mit Sebastian.
\end{itemize}

\section{Offenen Fragen und Aufgaben}
\begin{itemize}
\item Johannes updated grobe Struktur des Projektplanes und wei�t andere dazu auf ihre Aufgabengebiete genauer zu beschreiben
\item Treffen mit Sebastian.
\item Raum in der Bib reservieren (Johannes)
\item Kommenden Mittwochtermin mit Sebastian auf 12.45 Uhr verschieben.
\end{itemize}



\end{document}	